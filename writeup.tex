\documentclass[20pt,letterpaper]{article}
\usepackage[utf8]{inputenc}
\usepackage{physics}
\usepackage{fancyhdr}
\usepackage{amsmath}
\usepackage{cancel}
\usepackage{graphicx}
\usepackage{graphics}
\usepackage{extarrows}
\usepackage{caption}
\usepackage{subcaption}
\usepackage[margin=1in]{geometry}
\usepackage[version=4]{mhchem}
\usepackage{mathtools}
\usepackage{amsfonts}
\usepackage{amsmath}
\usepackage{amssymb}
\usepackage{xcolor}
\usepackage{sectsty}
\usepackage{ulem}
\usepackage{pgfplots}
\usepackage{mathrsfs}
\usepackage{pythonhighlight}
\newcommand*{\hatH}{\hat{\mathcal{H}}}
\newcommand*{\hatD}{\hat{\mathcal{D}}}
\pagestyle{fancy}
\fancyhf{}
\rhead{Physics 129L HW 5}
\lhead{Jin \thepage}

\begin{document}
\section{Problem 1}
Running Running the \textbf{ls -alF} command on the top level directory of the flash drive:\\
\textbf{Output:} included in the txt file: \textit{Problem\_1\_outputs\_a.txt}\\
Running the \textbf{ls -alF} command on the top level directory of the most recent backup on the flash drive:\\
\textbf{Output:} included in the txt file: \textit{Problem\_1\_outputs\_b.txt}
\section{Problem 2}
Using \textbf{diff} command on the python code use to solve Problem 3 and \textbf{img.py} we have the output:\\
included in the .txt file: \textit{Problem\_2\_outputs.txt}
\end{document}
